%%%%%%%%%%%%%%%%%%%%%%%%%%%%%%%%%%%%%%%%%%%%%%%%%%%%%%%%%%%%%%%%%%%%%%%%%%%%%%%%
%%%%%%%%%%%%%%% CAPA E PÁGINAS PRELIMINARES (TESE/DISSERTAÇÃO)  %%%%%%%%%%%%%%%%
%%%%%%%%%%%%%%%%%%%%%%%%%%%%%%%%%%%%%%%%%%%%%%%%%%%%%%%%%%%%%%%%%%%%%%%%%%%%%%%%

\usepackage{trimspaces}

% Formatação de datas de acordo com a língua
\usepackage[useregional]{datetime2}
\DTMusemodule{brazilian}{portuges}
\DTMnewdatestyle{month-year}{%
  \renewcommand*{\DTMdisplaydate}[4]{##2,\space##1}%
  \renewcommand*{\DTMDisplaydate}{\DTMdisplaydate}%
}

\makeatletter

%%%%%%%%%%%%%%%%%%%%%%%%%%%%%%%%%%%%%%%%%%%%%%%%%%%%%%%%%%%%%%%%%%%%%%%%%%%%%%%%
%%%%%%%%%%%%%%%%%%%%% TEXTOS PADRÃO EM PT E EN PARA A CAPA %%%%%%%%%%%%%%%%%%%%%
%%%%%%%%%%%%%%%%%%%%%%%%%%%%%%%%%%%%%%%%%%%%%%%%%%%%%%%%%%%%%%%%%%%%%%%%%%%%%%%%

% \extrasLANGUAGE vs \captionsLANGUAGE: https://tex.stackexchange.com/a/354197/217608

% Palavras fixas a serem traduzidas
\providecommand\keywordsname{} % Keywords / Palavras-chave
\providecommand\programname{} % Program / Programa
\providecommand\committeename{} % Examining committee / Comissão julgadora
\providecommand\advisorname{} % Advisor / Orientador(a)
\providecommand\coadvisorname{} % Co-advisor / Coorientador(a)
\providecommand\workname{} % Report, Thesis / Tese, Dissertação, Monografia
\providecommand\degreename{} % Masters, Doctorate, Bachelor / Mestrado, Doutorado, Bacharelado
\providecommand\titlename{} % Master, Doctor, Bachelor / Mestre(a), Doutor(a), Bacharel
\providecommand\@licenseboilerplate{O conteúdo deste trabalho é publicado sob a licença}

% Textos longos a serem traduzidos
\providecommand\@coverTCCText{}
\providecommand\@coverQualiText{}
\providecommand\@coverThesisText{}
\providecommand\@institutionBlockText{} % Só para TCC
\providecommand\@provisionalFrontmatterText{}
\providecommand\@finalFrontmatterText{}
\providecommand\@institution{}

% Este não precisa ser traduzido, o texto em inglês não utiliza
\providecommand\@bywhom{%
  \ifdefstring{\@authorGender}{masc}
    {pelo candidato \@author}
    {pela candidata \@author}%
}

%%%%%%%%%% PORTUGUÊS %%%%%%%%%%
\expandafter\addto\csname captions\@IMEpt\endcsname{%
  \DTMrenewdatestyle{month-year}{%
    \renewcommand*{\DTMdisplaydate}[4]
      {\DTMportugesmonthname{##2}\space de\space##1}%
  }%
  \let\@title\@titlept
  \let\@subtitle\@subtitlept
  \let\@keywords\@keywordspt
  \renewcommand\keywordsname{Palavras-chave}%
  \renewcommand\programname{Programa}%
  \renewcommand\committeename{Comissão julgadora}%
  \renewcommand\advisorname{%
    \iftoggle{@tcc}{%
      \ifdefstring{\@advisorGender}{masc}
        {Supervisor}
        {Supervisora}%
    }{%
      \ifdefstring{\@advisorGender}{masc}
        {Orientador}
        {Orientadora}%
    }%
  }%
  \renewcommand\coadvisorname[1]{%
    \iftoggle{@tcc}{%
      \ifcsstring{@coadvisor#1Gender}{masc}
        {Cossupervisor}
        {Cossupervisora}%
    }{%
      \ifcsstring{@coadvisor#1Gender}{masc}
        {Coorientador}
        {Coorientadora}%
    }%
  }%
  \renewcommand\workname{%
    \iftoggle{@tcc}
      {Monografia}
      {\iftoggle{@qualificacao}
        {Exame de Qualificação}
        {\iftoggle{@doutorado}
          {Tese}
          {Dissertação}%
        }%
      }%
  }%
  \renewcommand\degreename{%
    \iftoggle{@doutorado}
      {Doutorado}
      {\iftoggle{@mestrado}
        {Mestrado}
        {\iftoggle{@tcc}
          {Bacharelado}
          {Nível não definido!}%
        }%
      }%
  }%
  \renewcommand\titlename{%
    \iftoggle{@doutorado}
      {\ifdefstring{\@authorGender}{masc}{Doutor}{Doutora}}
      {\iftoggle{@mestrado}
        {\ifdefstring{\@authorGender}{masc}{Mestre}{Mestra}}
        {\iftoggle{@tcc}
          {Bacharel}{Nível não definido!}%
        }%
      }%
  }%
  %
  %
  \renewcommand\@coverTCCText{%
    Monografia Final\vspace{.5\baselineskip}\\
    \@macCDXCIX{} --- Trabalho de\\
    Formatura Supervisionado%
  }%
  \renewcommand\@coverQualiText{%
    Relatório apresentado ao\\
    Instituto de Matemática e Estatística\\
    da Universidade de São Paulo\\
    para exame de qualificação de\\
    \degreename{} em Ciências%
  }%
  \renewcommand\@coverThesisText{%
    \workname{} apresentada ao\\
    Instituto de Matemática e Estatística\\
    da Universidade de São Paulo\\
    para obtenção do título de\\
    \titlename{} em Ciências%
  }%
  \renewcommand\@institutionBlockText{%
    Universidade de São Paulo\\
    Instituto de Matemática e Estatística\\
    Bacharelado em Ciência da Computação%
  }%
  \renewcommand\@provisionalFrontmatterText{%
    \iftoggle{@qualificacao}{%
      Esta é a versão original do texto de qualificação elaborado
      \@bywhom{}, tal como submetido à Comissão Julgadora.%
    }{%
      Esta é a versão original da \MakeLowercase{\workname} elaborada
      \@bywhom{}, tal como submetida à Comissão Julgadora.%
    }%
  }%
  \renewcommand\@finalFrontmatterText{%
    Esta versão da \MakeLowercase{\workname} contém as correções e alterações
    sugeridas pela Comissão Julgadora durante a defesa da versão
    original do trabalho, realizada em \DTMusedate{@defensedate}.\\[1\baselineskip]
    Uma cópia da versão original está disponível no Instituto de
    Matemática e Estatística da Universidade de São Paulo.%
  }%
  \renewcommand\@institution{%
    Instituto de Matemática e Estatística,
    Universidade de São Paulo%
  }%
  \renewcommand\@licenseboilerplate{O conteúdo deste trabalho é publicado sob a licença}%
}


%%%%%%%%%% INGLÊS %%%%%%%%%%
\expandafter\addto\csname captions\@IMEen\endcsname{%
  \DTMrenewdatestyle{month-year}{%
    \renewcommand*{\DTMdisplaydate}[4]
      {\DTMenglishmonthname{##2},\space##1}%
  }%
  \let\@title\@titleen
  \let\@subtitle\@subtitleen
  \let\@keywords\@keywordsen
  \renewcommand\keywordsname{Keywords}%
  \renewcommand\programname{Program}%
  \renewcommand\committeename{Examining Committee}
  \renewcommand\advisorname{%
    \iftoggle{@tcc}{Supervisor}{Advisor}%
  }%
  \renewcommand\coadvisorname[1]{%
    \iftoggle{@tcc}{Co-supervisor}{Coadvisor}%
  }%
  % "Tese" e "dissertação" têm sentido contrário em língua inglesa:
  % http://guides.lib.berkeley.edu/dissertations_theses
  % https://www.grad.ubc.ca/handbook-graduate-supervision/graduate-thesis
  % Como "Thesis" é o nome genérico, vamos usar para mestrado e doutorado
  %
  %%%%%
  %
  % Nomes possíveis para o TCC em inglês:
  %
  % * monograph/monography
  %     usado para trabalho de alto nível de um autor "senior",
  %     então não faz sentido para um trabalho de graduação.
  %
  % * undergraduate thesis / bachelor's thesis
  %     plausível, mas no nosso caso report parece melhor.
  %
  % * senior project / senior thesis / honor thesis
  %     usado para "TCCs" de caráter fortemente acadêmico;
  %     não é o caso aqui.
  %
  % * essay / report
  %     razoável, porque trata-se de um texto/relato
  %     sobre o projeto de TCC.
  \renewcommand\workname{%
    \iftoggle{@tcc}
      {Capstone Project Report}
      {\iftoggle{@qualificacao}
        {Qualifying Exam}
        {Thesis}%
      }%
  }%
  \renewcommand\degreename{%
    \iftoggle{@doutorado}
      {Doctorate}
      {\iftoggle{@mestrado}
        {Master's}
        {\iftoggle{@tcc}
          {Bachelor}
          {Nível não definido!}%
        }%
      }%
  }%
  \renewcommand\titlename{%
    \iftoggle{@doutorado}
      {Doctor}
      {\iftoggle{@mestrado}
        {Master}
        {\iftoggle{@tcc}
          {Bachelor}%
          {Nível não definido!}%
        }%
      }%
  }%
  %
  %
  \renewcommand\@coverTCCText{%
    Final Essay\vspace{.5\baselineskip}\\
    \@macCDXCIX{} --- Capstone Project%
  }%
  \renewcommand\@coverQualiText{%
    Report presented to the\\
    Institute of Mathematics and Statistics\\
    of the University of São Paulo\\
    for the \titlename{} of Science\\
    qualifying examination\\%
  }%
  \renewcommand\@coverThesisText{%
    \workname{} presented to the\\
    Institute of Mathematics and Statistics\\
    of the University of São Paulo\\
    in partial fulfillment\\
    of the requirements\\
    for the degree of\\
    \titlename{} of Science%
  }%
  \renewcommand\@institutionBlockText{%
    University of São Paulo\\
    Institute of Mathematics and Statistics\\
    Bachelor of Computer Science%
  }%
  \renewcommand\@provisionalFrontmatterText{%
    \iftoggle{@qualificacao}{%
      This is the original version of the qualifying text prepared
      by candidate \@author, as submitted to the Examining Committee.%
    }{%
      This is the original version of the \MakeLowercase{\workname} prepared
      by candidate \@author, as submitted to the Examining Committee.%
    }%
  }%
  \renewcommand\@finalFrontmatterText{%
    This version of the \MakeLowercase{\workname} includes the corrections
    and modifications suggested by the Examining Committee during
    the defense of the original version of the work, which took
    place on \DTMusedate{@defensedate}.\\[1\baselineskip]
    A copy of the original version is available at the Institute of
    Mathematics and Statistics of the University of São Paulo.%
  }%
  \renewcommand\@institution{%
    Institute of Mathematics and Statistics,
    University of São Paulo%
  }%
  \renewcommand\@licenseboilerplate{The content of this work is published under the}%
}


%%%%%%%%%%%%%%%%%%%%%%%%%%%%%%%%%%%%%%%%%%%%%%%%%%%%%%%%%%%%%%%%%%%%%%%%%%%%%%%%
%%%%%%%%%%%%%%%%%%%%%%% COLETA E DEFINIÇÃO DE METADADOS %%%%%%%%%%%%%%%%%%%%%%%%
%%%%%%%%%%%%%%%%%%%%%%%%%%%%%%%%%%%%%%%%%%%%%%%%%%%%%%%%%%%%%%%%%%%%%%%%%%%%%%%%

\renewcommand\author[2][masc]{
  \gdef\@author{#2}
  \gdef\@authorGender{#1}
}

\NewDocumentCommand{\orientador}{O{masc} m}{
  \gdef\@advisor{#2}
  \gdef\@advisorGender{#1}
}

% Mais de um coorientador é raro, mas acontece
\ExplSyntaxOn
\newcounter{numberOfCoadvisors}
\NewDocumentCommand\coorientador{O{masc} m}{
    \stepcounter{numberOfCoadvisors}
    \tl_gclear_new:c {@coadvisor\Roman{numberOfCoadvisors}}
    \tl_gclear_new:c {@coadvisor\Roman{numberOfCoadvisors}Gender}

    \tl_set:cn {@coadvisor\Roman{numberOfCoadvisors}} {#2}
    \tl_set:cn {@coadvisor\Roman{numberOfCoadvisors}Gender} {#1}
}

\seq_gclear_new:N \@committeeMembers

\newtoggle{@mestrado}
\newtoggle{@doutorado}
\newtoggle{@tcc}
\newtoggle{@qualificacao}
\newtoggle{@finalversion}

% Opções usando LaTeX3 (veja texdoc l3keys).
\keys_define:nn { IME / defense }
  {
    % Chaves à esquerda definem as variáveis à direita
    data .code:n= {\DTMsavedate{@defensedate}{#1}},
    data .value_required:n = true,
    nivel .choice:,
    nivel / mestrado .code:n = {\@mestrado},
    nivel / masters .code:n = {\@mestrado},
    nivel / dissertacao .code:n = {\@mestrado},
    nivel / doutorado .code:n = {\@doutorado},
    nivel / phd .code:n = {\@doutorado},
    nivel / tese .code:n = {\@doutorado},
    nivel / graduacao .code:n = {\@tcc},
    nivel / bachelor .code:n = {\@tcc},
    nivel / tcc .code:n = {\@tcc},
    nivel .value_required:n = true,
    quali .code:n = {\ifstrequal{#1}{true}{\toggletrue{@qualificacao}}{\togglefalse{@qualificacao}}},
    quali .default:n = {true},
    definitiva .code:n = {\ifstrequal{#1}{true}{\toggletrue{@finalversion}}{\togglefalse{@finalversion}}},
    definitiva .default:n = {true},
    provisoria .code:n = {\ifstrequal{#1}{true}{\togglefalse{@finalversion}}{\toggletrue{@finalversion}}},
    provisoria .default:n = {true},
    programa .tl_gset:N = \@program,
    program .value_required:n = true,
    apoio .tl_gset:N = \@financing,
    apoio .value_required:n = true,
    local .tl_gset:N = \@defenselocation,
    local .value_required:n = true,
    direitos .tl_gset:N = \@license,
    direitos .value_required:n = true,
    fichacatalografica .tl_gset:N = \@catalogingData,
    fichacatalografica .value_required:n = true,
    membrobanca .code:n = {\seq_gput_right:Nn \@committeeMembers {#1}},
    membrobanca .value_required:n = true,
  }

\NewDocumentCommand\defesa{m}{%
  \keys_set:nn {IME/defense}{#1}

  \exp_args:NV \str_case:nnF \@license
    {
      {CC-BY}{\gdef\@license{\@licenseboilerplate\space CC~BY~4.0\\
        \href{https\c_colon_str//creativecommons.org/licenses/by/4.0/}{%
        (Creative~Commons~Attribution~4.0~International~License)}}
        \hypersetup{pdflicenseurl={https://creativecommons.org/licenses/by/4.0/}}
      }

      {CC-BY-NC}{\gdef\@license{\@licenseboilerplate\space CC~BY-NC~4.0\\
        \href{https\c_colon_str//creativecommons.org/licenses/by-nc/4.0/}{%
        (Creative~Commons~Attribution-NonCommercial~4.0~International~License)}}
        \hypersetup{pdflicenseurl={https://creativecommons.org/licenses/by-nc/4.0/}}
      }

      {CC-BY-ND}{\gdef\@license{\@licenseboilerplate\space CC~BY-ND~4.0\\
        \href{https\c_colon_str//creativecommons.org/licenses/by-nd/4.0/}{%
        (Creative~Commons~Attribution-NoDerivatives~4.0~International~License)}}
        \hypersetup{pdflicenseurl={https://creativecommons.org/licenses/by-nc-nd/4.0/}}
      }

      {CC-BY-SA}{\gdef\@license{\@licenseboilerplate\space CC~BY-SA~4.0\\
        \href{https\c_colon_str//creativecommons.org/licenses/by-sa/4.0/}{%
        (Creative~Commons~Attribution-ShareAlike~4.0~International~License)}}
        \hypersetup{pdflicenseurl={https://creativecommons.org/licenses/by-sa/4.0/}}
      }

      {CC-BY-NC-SA}{\gdef\@license{\@licenseboilerplate\space CC~BY-NC-SA~4.0\\
        \href{https\c_colon_str//creativecommons.org/licenses/by-nc-sa/4.0/}{%
        (Creative~Commons~Attribution-NonCommercial-ShareAlike~4.0~International~License)}}
        \hypersetup{pdflicenseurl={https://creativecommons.org/licenses/by-nc-sa/4.0/}}
      }

      {CC-BY-NC-ND}{\gdef\@license{\@licenseboilerplate\space CC~BY-NC-ND~4.0\\
        \href{https\c_colon_str//creativecommons.org/licenses/by-nc-nd/4.0/}{%
        (Creative~Commons~Attribution-NonCommercial-NoDerivatives~4.0~International~License)}}
        \hypersetup{pdflicenseurl={https://creativecommons.org/licenses/by-nc-nd/4.0/}}
      }
    }
    % If there is no match, use the user-supplied text
    {}
}

\seq_gclear_new:N \@seqkeywordspt
\seq_gclear_new:N \@seqkeywordsen
\newcommand*{\palavrachave}[1]{\seq_gput_right:Nn \@seqkeywordspt {#1}}
\newcommand*{\keyword}[1]{\seq_gput_right:Nn \@seqkeywordsen {#1}}

% Na impressão, as palavras-chave são separadas por pontos
\newcommand*{\@keywordspt}{\seq_use:Nn \@seqkeywordspt {.\space}.}
\newcommand*{\@keywordsen}{\seq_use:Nn \@seqkeywordsen {.\space}.}

% Para inclusão nos metadados com hyperxmp, são separadas por vírgulas
\newcommand*{\@commakeywordspt}{\seq_use:Nn \@seqkeywordspt {,}}
\newcommand*{\@commakeywordsen}{\seq_use:Nn \@seqkeywordsen {,}}

\ExplSyntaxOff

\NewDocumentCommand{\@doutorado}{}{
  \toggletrue{@doutorado}
  \togglefalse{@mestrado}
  \togglefalse{@tcc}
}

\NewDocumentCommand{\@mestrado}{}{
  \togglefalse{@doutorado}
  \toggletrue{@mestrado}
  \togglefalse{@tcc}
}

\NewDocumentCommand{\@tcc}{}{
  \togglefalse{@mestrado}
  \togglefalse{@doutorado}
  \toggletrue{@tcc}
}

% Defaults quando o usuário não define alguma dessas variáveis

\author{Autor não definido!}
\orientador{Orientador não definido!}
\DTMsavedate{@defensedate}{1970-01-01}
\providecommand\@program{Programa não definido!}
\providecommand\@financing{}
\providecommand\@defenselocation{Local não definido!}
\providecommand\@license{Direitos não definidos!}
\providecommand\@title{Título não definido!}
\providecommand\@titlept{Título em português não definido!}
\providecommand\@titleen{Título em inglês não definido!}
\providecommand\@shorttitle{título curto não definido!}
\providecommand\@resumo{Resumo não definido!}
\providecommand\@abstract{Abstract não definido!}


%%%%%%%%%%%%%%%%%%%%%%%%%%%%%%%%%%%%%%%%%%%%%%%%%%%%%%%%%%%%%%%%%%%%%%%%%%%%%%%%
%%%%%%%%%%%%%%%%%%%%%%%%%%%%%% TÍTULO E SUBTÍTULO %%%%%%%%%%%%%%%%%%%%%%%%%%%%%%
%%%%%%%%%%%%%%%%%%%%%%%%%%%%%%%%%%%%%%%%%%%%%%%%%%%%%%%%%%%%%%%%%%%%%%%%%%%%%%%%

\ExplSyntaxOn

% Opções usando LaTeX3 (veja texdoc l3keys).
\keys_define:nn { IME / title }
  {
    % Chaves à esquerda definem as variáveis à direita
    shorttitle .tl_gset:N = \@shorttitle,
    shorttitle .value_required:n = true,
    titlept .tl_gset:N = \@titlept,
    titlept .value_required:n = true,
    titleen .tl_gset:N = \@titleen,
    titleen .value_required:n = true,
    subtitlept .tl_gset:N = \@subtitlept,
    subtitlept .value_required:n = true,
    subtitleen .tl_gset:N = \@subtitleen,
    subtitleen .value_required:n = true,
  }

\RenewDocumentCommand\title{m}{
  \keys_set:nn {IME/title}{#1}

  % Ambos devem existir. Este é o default, mas o valor de fato é definido
  % por \captionsLANGUAGE.
  \ifdefvoid{\@titlept}
    {\let\@title\@titleen}
    {\let\@title\@titlept}

  % Estes talvez não existam, mas se um existe o outro deve existir também.
  % Este é o default, mas o valor de fato é definido por \captionsLANGUAGE.
  \ifdefvoid{\@subtitlept}
    {\let\@subtitle\@subtitleen}
    {\let\@subtitle\@subtitlept}

  \tl_if_blank:VT \@shorttitle
    {
      \let\@shorttitle\@title
      \@IMEremoveLinebreaksEtc{\@shorttitle}
    }
}

\ExplSyntaxOff


%%%%%%%%%%%%%%%%%%%%%%%%%%%%%%%%%%%%%%%%%%%%%%%%%%%%%%%%%%%%%%%%%%%%%%%%%%%%%%%%
%%%%%%%%%%%%%%%%%%%%%%%%%%%%%%%%%% DEDICATÓRIA %%%%%%%%%%%%%%%%%%%%%%%%%%%%%%%%%
%%%%%%%%%%%%%%%%%%%%%%%%%%%%%%%%%%%%%%%%%%%%%%%%%%%%%%%%%%%%%%%%%%%%%%%%%%%%%%%%

% A dedicatória vai em uma página separada, sem numeração,
% com o texto alinhado à direita e margens esquerda e
% superior muito grandes. Vamos fazer isso com uma minipage.
\newenvironment{dedicatoria} {
  \hypersetup{pageanchor=false} % Veja comentário em \maketitle

  \if@openright\cleardoublepage\else\clearpage\fi

  \thispagestyle{empty}
  \vspace*{140mm plus 0mm minus 100mm}
  \noindent
  \begin{FlushRight}
     \begin{minipage}[b][100mm][b]{100mm}
       \begin{FlushRight}
         \itshape
} {
       \end{FlushRight}
     \end{minipage}\hspace*{3em}
  \end{FlushRight}
  \vspace*{50mm plus 0mm minus 10mm}
  \if@openright\cleardoublepage\else\clearpage\fi

  \hypersetup{pageanchor=true}
}


%%%%%%%%%%%%%%%%%%%%%%%%%%%%%%%%%%%%%%%%%%%%%%%%%%%%%%%%%%%%%%%%%%%%%%%%%%%%%%%%
%%%%%%%%%%%%%%%%%%%%%%%%%%%%%%%%%%% RESUMO %%%%%%%%%%%%%%%%%%%%%%%%%%%%%%%%%%%%%
%%%%%%%%%%%%%%%%%%%%%%%%%%%%%%%%%%%%%%%%%%%%%%%%%%%%%%%%%%%%%%%%%%%%%%%%%%%%%%%%

% A página de resumo deve existir em português e inglês; ambas as versões
% utilizam o mesmo environment.

\NewDocumentCommand{\resumo}{+m}{\long\gdef\@resumo{#1}}
\DeclareDocumentCommand{\abstract}{+m}{\long\gdef\@abstract{#1}}

\newcommand\printResumoAbstract{
  \bgroup\bgroup % Dois grupos aninhados, veja a documentação da package babel
  \expandafter\selectlanguage\expandafter{\@IMEpt}
  \begin{IMEabstract}\@resumo\end{IMEabstract}
  \expandafter\selectlanguage\expandafter{\@IMEen}
  \begin{IMEabstract}\@abstract\end{IMEabstract}
  \egroup\egroup
}


\NewDocumentEnvironment{IMEabstract}{} {
  \if@openright\cleardoublepage\else\clearpage\fi
  \thispagestyle{empty}

    \begin{Center}\Large\bfseries\abstractname\end{Center}

  \vspace*{2em plus 1em minus 1em}

  \footnotesize

  % Esse é o jeito mais simples de mudar as margens de um parágrafo:
  % faz de conta que é uma lista
  \begin{list}{}{\rightmargin 4em \leftmargin 4em}
    \item\@selfReference
  \end{list}

  \vspace*{1em plus 1em minus 0em}
} {
  % Impede uma quebra de página entre esta linha e a próxima, ou seja,
  % entre a última linha do resumo/abstract e as palavras-chave.
  \@afterheading

  \vspace*{1em plus 1em minus .5em}

  \begingroup

      \setlength{\leftmargini}{\widthof{\textbf{\keywordsname:}\quad}}
      \setlength{\labelwidth}{\widthof{\textbf{\keywordsname:}}}
      \setlength{\labelsep}{\widthof{\quad}}

      \begin{description}\item[\keywordsname:]\@keywords\end{description}

  \endgroup
}


%%%%%%%%%%%%%%%%%%%%%%%%%%%%%%%%%%%%%%%%%%%%%%%%%%%%%%%%%%%%%%%%%%%%%%%%%%%%%%%%
%%%%%%%%%%%%%%%%%%%%%% IMPRIME A CAPA E A FOLHA DE ROSTO %%%%%%%%%%%%%%%%%%%%%%%
%%%%%%%%%%%%%%%%%%%%%%%%%%%%%%%%%%%%%%%%%%%%%%%%%%%%%%%%%%%%%%%%%%%%%%%%%%%%%%%%

\RenewDocumentCommand\maketitle{}{
  % Embora as páginas iniciais *pareçam* não ter numeração, a numeração
  % existe, só não é impressa. Os comandos \frontmatter, \mainmatter,
  % \pagenumbering etc. reiniciam a contagem de páginas quando os números
  % passam a ser impressos. Isso significa que há mais de uma página com
  % o número "1". O pacote hyperref não lida bem com essa situação, então
  % vamos desabilitar hyperlinks para números de páginas aqui.
  \hypersetup{pageanchor=false}
  \bgroup
  \onehalfspacing

  \@IMEcover
  \iftoggle{@tcc}{}{\@IMEtitlePage}
  \@IMEversoPage

  \egroup
  \if@openright\cleardoublepage\else\clearpage\fi
  \hypersetup{pageanchor=true}
}

% Layout da capa
\NewDocumentCommand{\@IMEcover}{} {
  \cleardoublepage
  \thispagestyle{empty}

  \begin{hyphenrules}{nohyphenation}
      \iftoggle{@tcc}{\@institutionBlock}{}
      \@titleBlock
      \vfill
      \@detailsBlock
  \end{hyphenrules}
}

% Layout para a página de rosto (duas versões, de acordo
% com a Resolução CoPGr 6018 de 13/10/2011)
\NewDocumentCommand{\@IMEtitlePage}{} {
  \if@openright\cleardoublepage\else\clearpage\fi
  \thispagestyle{empty}

  \begin{hyphenrules}{nohyphenation}
      \@titleBlock
      \vspace*{2cm plus 2cm minus 1cm}
      \@versionInfoBlock
      \vspace*{3.5cm plus 3cm minus 3.5cm}
      \iftoggle{@finalversion}{\@committeeBlock}{}
      \vspace*{2cm plus 2cm minus 2cm}
  \end{hyphenrules}
}

\NewDocumentCommand{\@IMEversoPage}{}{
  \clearpage
  \thispagestyle{empty}
  \vspace*{4cm plus 4cm minus 2cm}
  \@versoPageBlock
  \vspace*{8cm plus 5cm minus 6cm}
}


%%%%%%%%%%%%%%%%%%%%%%%%%%%%%%%%%%%%%%%%%%%%%%%%%%%%%%%%%%%%%%%%%%%%%%%%%%%%%%%%
%%%%%%%%%%%%%%%%%%%%%%%% POSIÇÃO DOS ELEMENTOS NA CAPA %%%%%%%%%%%%%%%%%%%%%%%%%
%%%%%%%%%%%%%%%%%%%%%%%%%%%%%%%%%%%%%%%%%%%%%%%%%%%%%%%%%%%%%%%%%%%%%%%%%%%%%%%%

% O IME usa uma capa padrão de cartolina para todas as teses/dissertações.
% Essa capa tem uma janela recortada por onde se vê o título e o autor do
% trabalho. Ela fica centralizada na página, tem 100m de largura, 60mm de
% altura e começa 47mm abaixo do topo da página. Como o documento já tem
% margens definidas pelo usuário, precisamos calcular quanto precisamos
% acrescentar ou subtrair dessas margens para colocar o título e autor
% na posição exata (na verdade, com uma pequena folga: 49mm abaixo do topo
% da página, 96mm de largura e 56mm de altura).
%
% Para centralizar horizontalmente, poderíamos pensar em usar "\center",
% mas isso não funciona porque ele centraliza o texto em relação à coluna
% de texto, não à página. Assim, como as margens esquerda e direita do
% documento podem ser diferentes, a janela não ficaria na posição correta.
% O que faremos, então, é colocar essa janela em uma minipage e calcular
% a margem esquerda para que essa minipage fique centralizada.
%
% Além disso, outros elementos da capa também não podem ser centralizados
% com "\center", porque eles ficariam desalinhados em relação à janela
% com o título e autor. Vamos colocar esses outros elementos em uma
% minipage também, mas de tamanho diferente da anterior.
%
% Então, precisamos calcular três valores: a margem adicional em relação ao
% topo da página, a margem esquerda da janela com título e autor e a margem
% esquerda para os demais elementos centralizados da página.

\AtEndPreamble{
  % Calcula o valor das margens (após geometry ser carregada)

  % A distância entre o topo da página e o início do texto (fora o cabeçalho)
  % é dada por (1in + \voffset + \headsep + \topmargin + \headheight).
  % Queremos colocar a caixa com o título 49mm abaixo do topo, então:
  \dimgdef\@topTitleBlockMargin{49mm - (1in + \voffset + \headsep + \topmargin + \headheight)}

  % Quando \vspace é usado no início da página, ele não tem efeito; como
  % não é isso que queremos, vamos usar \vspace*. No entanto, \vspace*
  % é implementado inserindo uma \hrule de espessura zero e depois
  % acrescentando o espaço solicitado. O resultado não é exatamente
  % o esperado, pois \topskip, \parskip e \baselineskip interagem com
  % \vspace* de maneira um tanto complexa:
  % https://tex.stackexchange.com/a/247516/183146
  %
  % Aqui, vamos compensar essa diferença. Note que, se a primeira linha
  % da página tivesse um tamanho de fonte especial, seria necessário
  % usar o valor de \baselineskip correspondente a essa fonte. Além
  % disso, definimos espaçamento simples porque o \vspace* mencionado
  % acima é executado com espaçamento simples.
  \bgroup
  \setstretch {\setspace@singlespace}% \singlespacing adds \baselineskip
  \dimgdef\@topTitleBlockMargin{\@topTitleBlockMargin - \baselineskip - \parskip}
  \egroup

  % Queremos colocar a caixa com o título centralizada na página. "\center"
  % centraliza em função da área de texto, não da página inteira, então
  % não podemos usá-lo, pois as margens esquerda e direita podem ser
  % diferentes. A distância entre a borda esquerda/interna do papel e o
  % início do texto é dada por (1in + \hoffset + \oddsidemargin), então:
  \dimgdef\@leftTitleBlockMargin{(\paperwidth - 96mm)/2 - (1in + \hoffset + \oddsidemargin)}
  \dimgdef\@coverLeftMargin{(\paperwidth - 160mm)/2 - (1in + \hoffset + \oddsidemargin)}
}


%%%%%%%%%%%%%%%%%%%%%%%%%%%%%%%%%%%%%%%%%%%%%%%%%%%%%%%%%%%%%%%%%%%%%%%%%%%%%%%%
%%%%%%%%%%%%% OS ELEMENTOS QUE COMPÕEM A CAPA E A FOLHA DE ROSTO %%%%%%%%%%%%%%%
%%%%%%%%%%%%%%%%%%%%%%%%%%%%%%%%%%%%%%%%%%%%%%%%%%%%%%%%%%%%%%%%%%%%%%%%%%%%%%%%

% Com fontspec (ou seja, lualatex/xelatex), o comando \oldstylenums funciona
% com qualquer fonte que tenha suporte a números old-style. Já com pdflatex,
% o comando para escolher números old style depende da fonte em uso. Nesse
% caso, se não soubermos qual a fonte atual (ou seja, não é nem libertine
% nem libertinus), vamos usar latin modern e torcer para o resultado não ser
% muito discrepante do restante do texto.

% 499 = CDXCIX
\@ifpackageloaded{fontspec}
  {\providecommand{\@macCDXCIX}{mac~\oldstylenums{499}}}
  {
    \providecommand{\@macCDXCIX}{{\fontfamily{lmr}\selectfont mac~\oldstylenums{499}}}

    \@ifpackageloaded{libertinus}
      {\renewcommand{\@macCDXCIX}{\LibertinusSerifOsF mac~499}}
      {}

    \@ifpackageloaded{libertine}
      {\renewcommand{\@macCDXCIX}{\libertineOsF mac~499}}
      {}
  }

\newcommand{\@coverText}{
  \bgroup
  \setstretch{.9}

  \iftoggle{@tcc}
    {\@coverTCCText}
    {\iftoggle{@qualificacao}{\@coverQualiText}{\@coverThesisText}}
  \par
  \egroup
}

\ExplSyntaxOn
\newcounter{@IMEtmpcnt}
\newcommand*{\@coverPeople} {%
  \begin{tabular}{rl}
    \iftoggle{@tcc}{}{\programname : & \@program \tabularnewline}
    \advisorname : & \@advisor \tabularnewline
    \setcounter{@IMEtmpcnt}{0}%
    \int_while_do:nNnn {\value{@IMEtmpcnt}} < {\value{numberOfCoadvisors}} {%
      \stepcounter{@IMEtmpcnt}%
      \coadvisorname{\Roman{@IMEtmpcnt}}: & \csuse{@coadvisor\Roman{@IMEtmpcnt}} \tabularnewline
    }%
  \end{tabular}
}
\ExplSyntaxOff

\newcommand{\@selfReference} {%
  \bgroup
  \@IMEremoveLinebreaksEtc{\@title}%
  \@IMEremoveLinebreaksEtc{\@subtitle}%
  \@IMEremoveLinebreaksEtc{\@author}%
  \@author.
  \textbf{\@title\ifdefvoid{\@subtitle}{}{: \textit{\@subtitle}}}.
  \workname{} (\degreename).
  \@institution,
  São Paulo, \DTMfetchyear{@defensedate}.%
  \egroup
}

\NewDocumentCommand{\@versoPageBlock}{} {
  \bgroup
  \onehalfspacing
  \begin{list}{}{\rightmargin 3em \leftmargin 3em}
    \item
      \bgroup\centering\footnotesize\itshape\@license\par\egroup

      \ifcsvoid{@catalogingData} {} {
        \vspace*{3cm plus 3cm minus 1cm}
        \setlength{\fboxsep}{20pt}
        \begin{Center}
        \fbox{
          \begin{minipage}[t]{120mm}
            \setlength\parskip{1em}

            \@catalogingData

          \end{minipage}
        }
        \end{Center}
      }
  \end{list}
  \egroup
}

% Só para TCC
\newcommand{\@institutionBlock}{

    % A posição do quadro de título é fixa em relação à página;
    % a posição deste quadro é definida em função da posição do
    % quadro de título. Assim, primeiro vamos encontrar onde
    % deve começar o quadro do título. Veja os comentários em
    % \@titleBlock para entender o mecanismo.
    \bgroup
    \setstretch {\setspace@singlespace}% \singlespacing adds \baselineskip

    \vspace*{\@topTitleBlockMargin}
    \ifdeflength{\@normalstrutheight}
      {}
      {\newlength{\@normalstrutheight}}
    \settoheight{\@normalstrutheight}{\strut}
    \vspace{-\@normalstrutheight}

    % Estamos alinhados com o quadro do título do trabalho,
    % mas não é isso que queremos: a parte inferior deste
    % quadro deve ficar 15mm acima do quadro de título e
    % este quadro tem 20mm de altura, então precisamos subir:
    \vspace{-20mm} % Espaço ocupado por este quadro
    \vspace{-15mm} % Espaço entre este quadro e o quadro de título

    \noindent\strut%
    \hspace*{\@coverLeftMargin}%
%    \fbox{%
      \begin{minipage}[t][20mm][s]{160mm}
        \vspace{0pt plus 20mm}

        \Centering\large

        \textsc{\@institutionBlockText}

        \vspace{0pt plus 20mm}
      \end{minipage}
%    }% fbox
    \par

    % Agora precisamos voltar o "cursor" para o começo da página
    % para que o quadro de título seja inserido no lugar certo.
    % Para isso, vamos:
    %
    % 1. Chegar novamente ao início do quadro de título e
    %
    % 2. Retroceder o tamanho da margem superior

    % compensa o espaço inserido por \par logo acima
    \vspace{-\parskip}
    \egroup

    % A altura da minipage já compensou o \vspace{-20mm} acima;
    % ainda precisamos compensar o \vspace{-15mm}
    \vspace{15mm}

    % Agora estamos no início do quadro de título, então
    % podemos recuar exatamente o tamanho da margem superior.
    \vspace{-\@topTitleBlockMargin}
}

% O quadro com o título e o autor que deve ser visível
% através da janela na capa.
\NewDocumentCommand{\@titleBlock}{} {

    \bgroup
    \setstretch {\setspace@singlespace}% \singlespacing adds \baselineskip

    % Este espaço coloca o topo da próxima linha
    % na posição que queremos:
    \vspace*{\@topTitleBlockMargin}

    % No entanto, a próxima linha contém apenas
    % uma minipage, e definir o topo de uma linha
    % desse tipo é complicado. Assim, vamos:
    %
    % 1. Acrescentar um \strut a essa linha;
    %
    % 2. mover o baseline dessa linha para o topo do \strut;
    %
    % 3. Alinhar o topo da minipage ao baseline da linha.
    %
    % Sobre alinhamento de minipages:
    % https://en.wikibooks.org/wiki/LaTeX/Boxes

    \ifdeflength{\@normalstrutheight}
      {}
      {\newlength{\@normalstrutheight}}
    \settoheight{\@normalstrutheight}{\strut}
    \vspace{-\@normalstrutheight}

    \noindent\strut
    \hspace*{\@leftTitleBlockMargin}%
%    \fbox{%
      \begin{minipage}[t][56mm][s]{96mm}
          \vspace*{2cm plus 1.5cm minus 1.8cm}

          \Centering\large

          \textbf{\@title}

          \vspace{0.3cm plus 0.2cm minus 0.1cm}

          \textbf{\textit{\@subtitle}}

          \vspace{1cm plus 1cm minus 0.6cm}

          \@author

          \vspace*{2cm plus 1.5cm minus 1.8cm}
      \end{minipage}%
%    }% fbox
    \par
    \egroup
}

% As demais informações da capa
\NewDocumentCommand{\@detailsBlock}{} {

  \bgroup
  \onehalfspacing
  \noindent
  \hspace*{\@coverLeftMargin}%
%  \fbox{%
    \begin{minipage}[t][130mm][s]{160mm}
      \begin{Center}
        \Large

        \vspace*{0.3cm plus 0.5cm minus 0.3cm}

        \textsc{\@coverText}

        \vspace*{1.5cm plus 0.5cm minus 0.5cm}

        \large\@coverPeople

        \vspace*{2.5cm plus 1cm minus 1cm}

        \normalsize

        \@financing

        \vspace*{1cm plus 1cm minus 0.3cm}

        \@defenselocation

        \iftoggle{@tcc}
          {\DTMfetchyear{@defensedate}}
          {\DTMsetdatestyle{month-year}\DTMusedate{@defensedate}}

      \end{Center}
    \end{minipage}%
%  }% fbox
  \par
  \egroup
}

% As informações da banca que vão apenas na versão definitiva
% da página de rosto
\ExplSyntaxOn
\NewDocumentCommand{\@committeeBlock}{} {
    \bgroup
    \onehalfspacing
    \begin{minipage}[t][][t]{\textwidth}
      \begin{quote}
        \normalsize\noindent\committeename :\par
        \begin{list}{}
        {
          \setlength{\leftmargin}{0pt}
          \setlength{\itemsep}{.1\baselineskip}
          \setlength{\topsep}{\baselineskip}
        }
          \item[] \seq_use:Nn \@committeeMembers {\item[]}
        \end{list}
      \end{quote}
    \end{minipage}
    \par
    \egroup
}
\ExplSyntaxOff

% A informação sobre a versão provisória ou definitiva
\NewDocumentCommand{\@versionInfoBlock}{} {%
  % As diretrizes dizem que "A natureza do trabalho, o grau pretendido, o
  % nome da instituição a que é submetido e a área de concentração devem
  % ser alinhados a partir do meio da parte impressa da página para a
  % margem direita, tanto na folha de rosto como na folha de avaliação."
  %
  % Assim, queremos alinhar o texto à direita com uma grande margem
  % à esquerda. Uma solução simples é alinhar o texto à direita
  % e inserir uma minipage. Dentro dela, definimos o texto
  % também alinhado à direita.

  \bgroup
  \onehalfspacing
  \begin{FlushRight}
    %\fbox{
      % Margem direita + 80mm de largura significa que a minipage
      % começa mais ou menos no meio da página.
      \begin{minipage}[t][50mm][s]{80mm}
        \begin{FlushRight}
          \normalsize
          \iftoggle{@finalversion}{%
            \@finalFrontmatterText%
          } {%
            \@provisionalFrontmatterText%
          }%
        \end{FlushRight}
      \end{minipage}
      \par
    %} % fbox
  \end{FlushRight}
  \egroup
}


%%%%%%%%%%%%%%%%%%%%%%%%%%%%%%%%%%%%%%%%%%%%%%%%%%%%%%%%%%%%%%%%%%%%%%%%%%%%%%%%
%%%%%%%%%%%%%%%%%%%%%%%%%%%%%% METADADOS XMP %%%%%%%%%%%%%%%%%%%%%%%%%%%%%%%%%%%
%%%%%%%%%%%%%%%%%%%%%%%%%%%%%%%%%%%%%%%%%%%%%%%%%%%%%%%%%%%%%%%%%%%%%%%%%%%%%%%%

% Insere os metadados XMP no arquivo PDF final. Alguns desses valores são
% definidos normalmente por hyperref/hyperxmp ou em hyperlinks.tex, mas
% para teses/dissertações vamos sobrescrevê-los aqui com AtEndPreamble.
% \@IMEremoveLinebreaksEtc está definida em hyperlinks.tex.
\AtEndPreamble{

  % Remove quebras de linha, notas de rodapé etc.
  \let\@cleantitleen\@titleen
  \let\@cleansubtitleen\@subtitleen
  \let\@cleantitlept\@titlept
  \let\@cleansubtitlept\@subtitlept
  \let\@cleanabstract\@abstract
  \let\@cleanresumo\@resumo

  \@IMEremoveLinebreaksEtc{\@cleantitleen}
  \@IMEremoveLinebreaksEtc{\@cleansubtitleen}
  \@IMEremoveLinebreaksEtc{\@cleantitlept}
  \@IMEremoveLinebreaksEtc{\@cleansubtitlept}
  \@IMEremoveLinebreaksEtc{\@cleanabstract}
  \@IMEremoveLinebreaksEtc{\@cleanresumo}

  \hypersetup{
    pdfauthor={\@author},
    % TODO: Seria ótimo apontar para uma licença, mas qual?
    %pdflicenseurl={https://creativecommons.org/licenses/by-nc-nd/4.0/},
  }

  % TODO: Com versões recentes de hyperxmp (final de 2020), não é
  %       recomendado definir pdflang; no futuro, isto deve ser mudado.
  \IfLanguagePatterns{brazilian}
    {
      \hypersetup{
        pdflang={pt},
        pdfmetalang={pt},
        pdftitle={\@cleantitlept\ifdefvoid{\@cleansubtitlept}{}{: \@cleansubtitlept}},
        pdfsubject={\@cleanresumo},
        pdfkeywords={\@commakeywordspt},
      }
      % XMPLangAlt redefines "\do"; this may cause
      % problems with biblatex, so let's use a group.
      % https://github.com/plk/biblatex/issues/1105
      \bgroup
      \XMPLangAlt{en}{pdfsubject={\@cleanabstract}}
      \XMPLangAlt{en}{pdftitle={\@cleantitleen\ifdefvoid{\@cleansubtitleen}{}{: \@cleansubtitleen}}}
      \egroup
      % o item "keywords" não pode ser traduzido
    }
    {
      \hypersetup{
        pdflang={en},
        pdfmetalang={en},
        pdftitle={\@cleantitleen\ifdefvoid{\@cleansubtitleen}{}{: \@cleansubtitleen}},
        pdfsubject={\@cleanabstract},
        pdfkeywords={\@commakeywordsen},
      }
      % XMPLangAlt redefines "\do"; this may cause
      % problems with biblatex, so let's use a group.
      % https://github.com/plk/biblatex/issues/1105
      \bgroup
      \XMPLangAlt{pt}{pdfsubject={\@cleanresumo}}
      \XMPLangAlt{pt}{pdftitle={\@cleantitlept\ifdefvoid{\@cleansubtitlept}{}{: \@cleansubtitlept}}}
      \egroup
      % o item "keywords" não pode ser traduzido
    }
}


%%%%%%%%%%%%%%%%%%%%%%%%%%%%%%%%%%%%%%%%%%%%%%%%%%%%%%%%%%%%%%%%%%%%%%%%%%%%%%%%
%%%%%%%%%%%%%%%%%%%%%%%%%%%%% SUMÁRIO E SEÇÕES %%%%%%%%%%%%%%%%%%%%%%%%%%%%%%%%%
%%%%%%%%%%%%%%%%%%%%%%%%%%%%%%%%%%%%%%%%%%%%%%%%%%%%%%%%%%%%%%%%%%%%%%%%%%%%%%%%

% Coloca as linhas "Apêndices" e "Anexos" no sumário. Com a opção "inline",
% cada apêndice/anexo aparece como "Apêndice X" ao invés de apenas "X".
\usepackage{appendixlabel} % carregado do diretório extras (veja basics.tex)

% titlesec permite definir formatação personalizada de títulos, seções etc.
% Observe que titlesec é incompatível com os comandos refsection
% e refsegment do pacote biblatex!
% Esta package utiliza titlesec e implementa a possibilidade de incluir
% uma imagem no título dos capítulos com o comando \imgchapter (leia
% os comentários no arquivo da package).
\usepackage{imagechapter} % carregado do diretório extras (veja basics.tex)

\makeatother
