%!TeX root=../tese.tex
%("dica" para o editor de texto: este arquivo é parte de um documento maior)
% para saber mais: https://tex.stackexchange.com/q/78101/183146

%% ------------------------------------------------------------------------- %%
\chapter{Introdução}
\label{cap:introducao}

Com a variedade de serviços oferecidos pela internet, a quantidade de requisições
que são feitas a um servidor web aumenta. De acordo com \cite{facebook:topProblems}, o Facebook em 2014 recebia
uma quatidade de consultas da ordem de bilhões por segundo. Nesse sentido é esperado que
também aumente o número de requisições maliciosas, como por exemplo SQL injection que está 
no top 10 da OWASP \cite{owasp:top10}, uma organização sem fins lucrativos que visa melhorar a segurança em 
softwares, através da publicação de artigos, metodologias e documentação de maneira gratuita.

Em tal cenário se faz necessário um sistema automático de detecção de requisições maliciosas, pois analisar cada
uma delas manualmente pode ser inviável ou bastante custoso. Uma possível solução é criar modelos de aprendizagem
de máquina para encontrar padrões de requisições maliciosas e então tomar decisões automatizadas com base no
resultado desses modelos.

Além disso, é desejável que esse projeto tenha um custo baixo em termos financeiros, de rede e
de consumo de energia. Para isso, se tem proposto o uso de hardware de baixo custo para implantação 
desse tipo de sistema. Em \cite{sbrc_estendido:lucas} há uma análise que conclui que é possível usar Raspberry Pi como nó de
um cluster de processamento de dados.

\subsection{Objetivo}

O objetivo deste conclusão de curso é construir modelos de aprendizagem de máquina para classificar
requisições maliciosas. Tais modelos utilizam logs de servidores HTTP para detectar potenciais ataques
de SQL Injection e XSS. E por fim, rodar esses modelos em Raspberry Pi.

Portanto, avançar o que foi realizado no TCC “Análise de Desempenho de Computadores de Baixo Custo em um Sistema de 
Detecção de Intrusão” de Lucas Seiki Oshiro \cite{tcc:lucas}.

\subsection{Resultado alcançados}

Aqui vem os resultados alcançados

\subsection{Organização da monografia}

Os próximos capítulos desta monografia estão dividos da seguinte maneira: 

\begin{itemize}
    \item Capítulo 2: Trabalhos relacionados a este TCC e resultados encontrados.
    \item Capítulo 3: Conceitos e ferramentas que foram usados.
    \item Capítulo 4: Ambiente usado e como os dados foram coletados.
    \item Capítulo 5: Como o modelo final foi selecionado.
    \item Capítulo 6: Metodologia e os resultados dos experimentos.
    \item Capítulo 7: Conclusão.
    \item Capítulo 8: Avaliação pessoal e crítica.
\end{itemize}