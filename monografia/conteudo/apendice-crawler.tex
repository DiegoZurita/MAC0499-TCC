%!TeX root=../tese.tex
%("dica" para o editor de texto: este arquivo é parte de um documento maior)
% para saber mais: https://tex.stackexchange.com/q/78101/183146

\chapter{Crawler}
\label{ap:crawler}

O Crawler pode ser visto em \url{https://github.com/DiegoZurita/MAC0499-TCC/blob/main/projeto/scripts/crawler.py}.
Ele foi utilizado para simular as requisições de acordo com o que foi explicado na seção 4.3.1.

O método execRequest realiza uma requisição com os parâmetros necessários para o DVWA. 
O método getRandomString gera uma palavra aleatória que será usada como parâmetros na requisição.
E por fim no método createRequest, é gerado um conjunto de requisições tal que certas urls sejam mais
requisitadas do que outras



