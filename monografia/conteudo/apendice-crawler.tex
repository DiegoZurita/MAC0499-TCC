%!TeX root=../tese.tex
%("dica" para o editor de texto: este arquivo é parte de um documento maior)
% para saber mais: https://tex.stackexchange.com/q/78101/183146

\chapter{Crawler}
\label{ap:crawler}


\begin{lstlisting}[language=Python]
  import sys
  from multiprocessing import Pool
  from dataclasses import dataclass
  import numpy as np
  from typing import List
  import string
  import requests
  from fake_useragent import UserAgent
  from time import sleep
  
  
  @dataclass
  class RequestData:
      host: str
      url: str
      sessid: str
      userAgent: str
  
  
  def execRequest(requestData: RequestData) -> bool:
      print("Requeisitando url {}{} ".format(requestData.host, requestData.url))
      headers = {
          'User-Agent': requestData.userAgent
      }
      cookies = {
          "PHPSESSID": requestData.sessid,
          "security": "low"
      }
      req = requests.get(
          "{}{}".format(requestData.host, requestData.url), 
          headers = headers,
          cookies = cookies
      )
      
      sleep(0.05)
  
      return req.status_code == 200
  
  
  def genRandomString():
      qtd_extra = np.random.randint(20, 60)
      random_string_list = np.random.choice(list(string.ascii_letters), qtd_extra)
      return  "".join(random_string_list)
  
  def createRequests(host: str, urls: List[str], n: int, sessid: str) -> List[RequestData]:
      url_index = np.random.binomial(len(urls), 0.5, n)
      qtdExtraPrams = np.random.choice([0, 1, 2, 3], n)
      ua = UserAgent()
  
      requests = []
  
      for i in range(n):
          url = urls[url_index[i]]
  
          if qtdExtraPrams[i] == 1:
              random_string = genRandomString()
  
              url = "{}?extraParam={}".format(url, random_string)
          elif qtdExtraPrams[i] == 2:
              random_string = genRandomString()
              random_string2 = genRandomString()
  
              url = "{}?extraParam={}&extraParam2={}".format(url, random_string, random_string2)
  
          
          requests.append(RequestData(host, url, sessid, ua.random))
  
      return requests
  
  def main():
      urls = [
          "/instructions.php",
          "/setup.php",
          "/vulnerabilities/brute/",
          "/vulnerabilities/exec/",
          "/vulnerabilities/csrf/",
          "/vulnerabilities/upload/",
          "/vulnerabilities/captcha/",
          "/vulnerabilities/sqli/",
          "/index.php",
          "/about.php",
          "/vulnerabilities/sqli_blind/",
          "/vulnerabilities/weak_id/",
          "/vulnerabilities/xss_d/",
          "/vulnerabilities/xss_r/",
          "/vulnerabilities/xss_s/",
          "/unknown", ## papel de fazer 404
          "/vulnerabilities/csp/",
          "/vulnerabilities/javascript/",
          "/security.php"
      ]
  
      host = sys.argv[1]
      n_requests = int(sys.argv[2])
      sessId = sys.argv[3]
      reqs = createRequests(host, urls, n_requests, sessId)
  
      with Pool(8) as p:
          results = p.map(execRequest, reqs)
  
          print("Porcentagem de requisições 200: {}".format(np.sum(results)/n_requests))
  
  if __name__ == "__main__":
      main()
\end{lstlisting}