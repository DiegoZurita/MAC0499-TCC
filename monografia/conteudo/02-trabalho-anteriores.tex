%!TeX root=../tese.tex
%("dica" para o editor de texto: este arquivo é parte de um documento maior)
% para saber mais: https://tex.stackexchange.com/q/78101/183146

%% ------------------------------------------------------------------------- %%
\chapter{Revisão da literatura}
\label{cap:fundamentation}

O uso de modelos de aprendizagem de máquina para detectar intrusões em servidores HTTP 
por meio de logs já foi estudado em outros trabalhos. Abaixo cito os artigos que nos
serviram de referência e qual modelo foi utilizado.

\begin{itemize}
    \item \cite{ref:art3}: regras estabelecidas manualmente após uma análise dos logs,
    isto é, nenhum modelo de aprendizagem de máquina tradicional foi usado.
    \item \cite{ref:art4}: árvores de decisão e SVM.
    \item \cite{ref:art6}: árvores de decisão e redes neurais.
    \item \cite{ref:art2}: naive bayes.
    \item \cite{ref:art7}: redes neurais.
    \item \cite{ref:art1}: árvores de decisão e, assim como o primeiro item, utilizou 
    regras pre estabelecidas.
\end{itemize}

Observar estes trabalhos direcionou a pesquisa no sentido de qual modelo avaliar, além de como 
e quais dados foram utilizados.

Vale deixar registrado, um agradecimento ao autor Jaron Fontaine do trabalho \cite{ref:art6}, que gentilmente
indicou quais logs reais foram usados em seu trabalho e onde encontrá-los.