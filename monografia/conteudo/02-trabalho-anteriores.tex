%!TeX root=../tese.tex
%("dica" para o editor de texto: este arquivo é parte de um documento maior)
% para saber mais: https://tex.stackexchange.com/q/78101/183146

%% ------------------------------------------------------------------------- %%
\chapter{Revisão da literatura}
\label{cap:fundamentation}

O uso de modelos de aprendizagem de máquina para detectar intrusões em servidores HTTP 
por meio de logs já foi estudado em outros trabalhos. Abaixo citamos os artigos que nos
serviram de referência e qual modelo foi utilizado.

\begin{itemize}
    \item \cite{ref:art3}: este trabalho analisa uma fase anterior dos ataques HTTP, a etapa
    de exploração de vulnerabilidades. Nessa etapa, em geral, um crawler pode ser executado 
    em busca de possíveis vulnerabilidades, e nesse sentido o processo distinguir tais 
    requisições das requisições de usuários ajuda na prevenção de ataques. Para isso, o 
    trabalho propôs regras estabelecidas manualmente após uma análise dos logs, isto é, 
    nenhum modelo de aprendizagem de máquina tradicional foi usado.

    \item \cite{ref:art6}: este trabalho visa detectar atividades maliciosas no ambiente 
    da nuvem. Uma vez que sua adoção vem crescendo, a superfície de ataque cresce proporcionalmente.
    Nesse sentido, o trabalho propõem analisar arquivos de logs gerados pela nuvem e por aplicações
    web para treinar modelos de aprendizagem de máquina e automatizar a sua detecção. Para isso
    usou modelo de árvore de decisão e redes neurais.

    \item \cite{ref:art2}: similar ao trabalho anterior, este trabalho propõe automatizar
    a detecção de ataques utilizando aprendizagem de máquina, contudo, neste trabalho
    o foco principal é o SQL injection motivado pela grave falta de privacidade 
    que esta falha pode proporcionar. Para isso propõe um modelo de naive bayes.
\end{itemize}

Observar estes trabalhos direcionou a pesquisa no sentido de que os modelos de 
aprendizagem de máquina podem ser utilizados para detectar falhas de segurança
em servidores HTTP, além disso qual modelo avaliar, e por fim quais dados foram utilizados.

Vale deixar registrado, um agradecimento ao autor Jaron Fontaine do trabalho \cite{ref:art6}, que gentilmente
indicou quais logs reais foram usados em seu trabalho e onde encontrá-los.