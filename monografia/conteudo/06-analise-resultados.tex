%!TeX root=../tese.tex
%("dica" para o editor de texto: este arquivo é parte de um documento maior)
% para saber mais: https://tex.stackexchange.com/q/78101/183146

%% ------------------------------------------------------------------------- %%
\chapter{Análise de resultados}
\label{cap:results}

Este capítulo apresenta a metodologia e resultados dos experimentos, como também 
expõe a sua análise.

\section{Metodologia}

Dois experimentos foram feitos: o primeiro para verificar a viabilidade do uso 
de Raspberry Pi para treinar modelos e o segundo usando-o somente para realizar as
predições. Em ambos os casos, apenas o modelo de árvore de decisão foi usado.

Para o experimento de treino, a seguinte metodologia foi usada:

\begin{itemize}
    \item Configuração do Apache Spark nos computadores.
    \item Início do script raspberry-log. 
    \item Rodar 20 vezes o script de treino e obtém os seus tempos.
    \item Fim do script raspberry-log. 
    \item Coleto as informações obtidas: tempo, consumo de memória e CPU.
\end{itemize}

Para avalisar o desempenho na tarefa de predicões, três subconjuntos
dos logs reais foram usados: com 1000, 5000 e 10000 linhas. E para isso a seguinte 
metodologia foi usada: 

\begin{itemize}
    \item Configuração do Apache Spark nos computadores.
    \item Início do script raspberry-log. 
    \item Rodar 20 vezes o script de predição usando a entrada respectiva e obtem os seus tempos.
    \item Fim do script raspberry-log. 
    \item Coleto as informações coletas: tempo, consumo de memória e CPU.
\end{itemize}

Os passos de ambos os casos foram executados no Macbook e no Raspberry Pi.

O script raspberry-log foi reaproveitado do trabalho \cite{tcc:lucas}. Ele foi usado para 
coletar o consumo de memória e CPU. Para coletar o tempo, o utilitário time 
foi usado.


\section{Experimentos}

\subsection{Controle}

O objetivo deste experimento foi medir o consumo de memória e CPU dos computadores
em repouso. Para isso o script raspberry-log foi executado por aproximademente 10 horas, 
para termos valores mais confiáveis. Vale ressaltar que nenhum outro programa estava sendo 
executado, apenas o nosso script.

O macbook em repouso consumiu 5.4\% de CPU com um desvio padrão de 0.6 
e 679.34MB de memória com um 30 de desvio padrão.

O Raspberry Pi em repouso consumiu 0.2\% de CPU com um desvio padrão de 0.05 
e 51.9MB de memória com um 0.59 de desvio padrão.

Os valores acima foram considerados estáveis, portanto considerados como os valores
do computador em repouso.



\subsection{Treinamento}

Aqui apresentamos os resultados do experimento onde usar o computador para treinar 
modelos é utilizado. Aqui um conjunto de dados de aproximademente 7000 linhas foi 
usado para treinar e 3000 para validar o modelo.

O resumo dos valores coletados estão na tabela \ref{tab:comparativo_treino}, entre parentesis está desvio
padrão.

\begin{table}[!ht]
    \centering
    \begin{tabular}{|l|l|l|}
    \hline
        ~ & Macbook & Raspberry Pi \\ \hline
        CPU & 22.58\% (11.626) & 41.42\% (11.23) \\ \hline
        RAM & 1036.05MB (34.52) & 270.94MB (82.28) \\ \hline
        Tempo & 18.54s (0.16) & 110.96s (2.12) \\ \hline
    \end{tabular}

    \caption{Comparativo métricas no experimento de treino.\label{tab:comparativo_treino}}
\end{table}

Note que o tempo médio de treinamento do Raspberry Pi é aproximademente 10 vezes maior do que no Macbook,
contudo isso não inviabiliza seu uso para treinar modelos, considerando que nesta etapa é natural
esperar que treinar modelos em grande quantidades de dados leve um tempo consideravel.

Observa-se também que o consumo de CPU do Raspberry Pi foi consideravelmente maior do que o observado 
em repouso, quando comparado ao Macbook. Nesse sentido, o treino deve ser feito preferencialmente 
em um computador com configurações similares ao do Macbook utilizado.


\subsection{Classificação}

Este experimento apenas realiza a classificação dos dados usando um modelo previamente treinado, 
da seguinte maneira:

\begin{itemize}
    \item Trienamos o modelo previamente e salvamos no disco.
    \item Outro script carrega esse modelo na memória. 
    \item Com o modelo na memória, apenas a classificação é executada.
\end{itemize}


Os resultados desse experimento estão nas tabelas abaixo:


\begin{table}[!ht]
    \centering
    \begin{tabular}{|l|l|l|}
    \hline
        ~ & Macbook & Raspberry Pi \\ \hline
        CPU & 21.62\% (12.01) & 39.69\% (11.80) \\ \hline
        RAM & 888.28MB (62.50) & 250.46MB (77.83) \\ \hline
        Tempo & 15.93s (0.16) & 84.87s (2.03) \\ \hline
    \end{tabular}

    \caption{Comparativo da classificação de 1000 linhas.\label{tab:comparativo_classificacao_1000}}
\end{table}


\begin{table}[!ht]
    \centering
    \begin{tabular}{|l|l|l|}
    \hline
        ~ & Macbook & Raspberry Pi \\ \hline
        CPU & 20.76\% (10.36) & 39.29\% (11.46) \\ \hline
        RAM & 943.36MB (48.83) & 256.45MB (78.85) \\ \hline
        Tempo & 16.30s (0.14) & 91.72s (3.06) \\ \hline
    \end{tabular}

    \caption{Comparativo da classificação de 5000 linhas.\label{tab:comparativo_classificacao_5000}}
\end{table}


\begin{table}[!ht]
    \centering
    \begin{tabular}{|l|l|l|}
    \hline
        ~ & Macbook & Raspberry Pi \\ \hline
        CPU & 21.09\% (10.74) & 39.06\% (11.78) \\ \hline
        RAM & 981.73MB (43.44) & 257.01MB (80.69) \\ \hline
        Tempo & 16.53s (0.15) & 93.22s (0.17) \\ \hline
    \end{tabular}

    \caption{Comparativo da classificação de 10000 linhas.\label{tab:comparativo_classificacao_10000}}
\end{table}


Destas tabelas, observa-se que o tempo médio de execução do Raspberry Pi ainda é maior 
do que o do Macbook, classificando em média 604 logs/s e 107 logs/s, respectivamente. Isto é,
o primeiro deve demorar, em média, 6 vezes mais para realizar classificações do que um computador tradicional 
com as configurações similares ao do Macbook utilizado, o que pode ser um impeditivo dependendo do 
volume de dados e da urgência em obter os resultados.
