%!TeX root=../tese.tex
%("dica" para o editor de texto: este arquivo é parte de um documento maior)
% para saber mais: https://tex.stackexchange.com/q/78101/183146

%% ------------------------------------------------------------------------- %%
\chapter{Avaliação pessoal e crítica}
\label{cap:avaliacao_pessoal}

O tema deste trabalho foi escolhido principalmente por dois motivos:

\begin{itemize}
    \item Aprofundar no tema de aprendizagem de máquina.
    \item Aprender mais sobre computação distribuída e Apache Spark.
\end{itemize}

O primeiro foi atingido, pois o contato e desenvolvimento nesse tema que eu tive
foi muito produtivo e me ajudou em muito.

O segundo, foi atendido parcialmente, dado que o contato com Apache Spark ocorreu,
contudo, durante os experimentos, o segundo Raspberru Pi queimou e a oportunidade de 
trabalhar com aprendizagem de máquina de maneira distribuída não ocorreu como eu 
esperava. Porém fica o aprendizado de estar preparado para qualquer adversidade em
trabalhos futuros.

A crítica principal a este trabalho foi ter utilizado o Apache Spark em um único nó 
de Raspberry Pi, pois esse não é o objetivo da ferramenta, contudo o experimento e 
script já estavam prontos e alterá-los nos tomaria tempo que não dispunhamos. 

Por fim, durante o trabalho o conhecimento das seguintes disciplinas foi utilizado: sistemas
operacionais para entender como utilizar da melhor maneira cada computador, ciência e engenharia
de dados que mostrou ferramentas para processamento de dados, aprendizagem de máquina onde os 
modelos foram expostos de maneira mais formal, língua portuguesa para a escrita desta monografia
e as disciplinas de estatística para entender melhor os modelos, além de ajudar a formalizar os 
experimentos. Por fim, a prática adquirida no grupo de extensão  BeeData, um dos vários grupos 
de extensão do IME-USP, mostrou-se bastante útil neste trabalho.
