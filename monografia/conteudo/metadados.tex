%!TeX root=../tese.tex
%("dica" para o editor de texto: este arquivo é parte de um documento maior)
% para saber mais: https://tex.stackexchange.com/q/78101/183146

%%%%%%%%%%%%%%%%%%%%%%%%%%%%%%%%%%%%%%%%%%%%%%%%%%%%%%%%%%%%%%%%%%%%%%%%%%%%%%%%
%%%%%%%%%%%%%%%%%%%%%%%%%%%%% METADADOS DA TESE %%%%%%%%%%%%%%%%%%%%%%%%%%%%%%%%
%%%%%%%%%%%%%%%%%%%%%%%%%%%%%%%%%%%%%%%%%%%%%%%%%%%%%%%%%%%%%%%%%%%%%%%%%%%%%%%%

% Estes comandos definem o título e autoria do trabalho e devem sempre ser
% definidos, pois além de serem utilizados para criar a capa, também são
% armazenados nos metadados do PDF.
\title{
    % Obrigatório nas duas línguas
    titlept={Classificação de requisições HTTP maliciosas por meio de aprendizagem de máquina},
    titleen={Classification of malicious HTTP requests through machine learning},
}

\author[masc]{Diego Ignacio Zurita Rojas}

% Para TCCs, este comando define o supervisor
\orientador[masc]{Profª. Dr. Daniel Macêdo Batista}


% A página de rosto da versão para depósito (ou seja, a versão final
% antes da defesa) deve ser diferente da página de rosto da versão
% definitiva (ou seja, a versão final após a incorporação das sugestões
% da banca).
\defesa{
  nivel=tcc, % mestrado, doutorado ou tcc
  % É a versão para defesa ou a versão definitiva?
  %definitiva,
  % É qualificação?
  %quali,
  programa={Ciência da Computação},
  membrobanca={Profª. Drª. Fulana de Tal (orientadora) -- IME-USP [sem ponto final]},
  % Em inglês, não há o "ª"
  %membrobanca{Prof. Dr. Fulana de Tal (advisor) -- IME-USP [sem ponto final]},
  membrobanca={Prof. Dr. Ciclano de Tal -- IME-USP [sem ponto final]},
  membrobanca={Profª. Drª. Convidada de Tal -- IMPA [sem ponto final]},
  local={São Paulo},
  data=2021-12-12, % YYYY-MM-DD
  % A licença do seu trabalho. Use CC-BY, CC-BY-NC, CC-BY-ND, CC-BY-SA,
  % CC-BY-NC-SA ou CC-BY-NC-ND para escolher a licença Creative Commons
  % correspondente (o sistema insere automaticamente o texto da licença).
  % Se quiser estabelecer regras diferentes para o uso de seu trabalho,
  % converse com seu orientador e coloque o texto da licença aqui, mas
  % observe que apenas TCCs sob alguma licença Creative Commons serão
  % acrescentados ao BDTA.
  direitos={CC-BY}, % Creative Commons Attribution 4.0 International License
  %direitos={Autorizo a reprodução e divulgação total ou parcial
  %          deste trabalho, por qualquer meio convencional ou
  %          eletrônico, para fins de estudo e pesquisa, desde que
  %          citada a fonte.},
  % Isto deve ser preparado em conjunto com o bibliotecário
  %fichacatalografica={nome do autor, título, etc.},
}

% As palavras-chave são obrigatórias, em português e
% em inglês. Acrescente quantas forem necessárias.
\palavrachave{Aprendizagem de máquina}
\palavrachave{Logs}
\palavrachave{Segurança Computacional}
\palavrachave{Redes de Computadores}


\keyword{Machine learning}
\keyword{Logs}
\keyword{Computational Security}
\palavrachave{Computer Networks}

% O resumo é obrigatório, em português e inglês.
\resumo{
Com o grande volume de requisições a websites, o número de de requisições maliciosas 
que se aproveitam de vulnerabilidades aumenta propocionalmente. Nesse sentido, se faz 
necessária a automação de detecções de requisições maliciosas, já que a verificação manual 
nem sempre consegue ser rápida o suficiente para impedir a exploração das vulnerabilidades. 
Este trabalho estuda modelos de aprendizagem de máquina para tal automação por meio da análise 
de logs de servidor HTTP. Além disso, analisa seu desempenho em um Raspberry Pi.
}

\abstract{
As large volume of requests to websites increase, the number of attacks also
increases proportionately. In this scenario, it is necessary to automate detections
of intrusions. This work studies machine learning models for
such automation through the analysis of HTTP server logs. Furthermore,
analyzes its performance on a Raspberry Pi. 
}
